\documentclass[11pt]{article}
\usepackage[a4paper, margin=2.5cm]{geometry}

\usepackage[T1]{fontenc}        % kodiranje znakov v .pdf
\usepackage[utf8]{inputenc}     % kodiranje znakov v .tex
\usepackage[slovene]{babel}  

\usepackage{amsthm} 
\usepackage{graphicx}

\usepackage{amsmath}
\newcommand{\f}{\mathcal{F}} 

% Brownovo gibanje
% Matej Rojec

\title{Brownovo gibanje}
\author{Matej Rojec}
\date{}

\begin{document}

\maketitle

\theoremstyle{plain}
\newtheorem{izrek}{Izrek}

\theoremstyle{definition}
\newtheorem{definicija}{Definicija}

Brownovo gibanje (več v \cite{karatzas1991brownian}) je intuitivno slučajen proces, % Sklic na knjigo
ki predstavlja naključno gibanje delcev v mediju.
    
    % Slika: PerrinPlot2.pdf
    % Napis pod sliko: 
    % Reprodukcija slike iz Jean Baptiste Perrin, \emph{Mouvement brownien et réalité moléculaire}, Ann. de Chimie et de Physique (VIII) 18, 5-114, 1909

\begin{figure}[!ht]
    \centering
    \caption{Reprodukcija slike iz Jean Baptiste Perrin, \emph{Mouvement brownien et réalité moléculaire}, Ann. de Chimie et de Physique (VIII) 18, 5-114, 1909};
    \includegraphics[scale=0.5]{PerrinPlot2.pdf}
\end{figure}

    % Začetek definicije
    \begin{definicija}
        Standardno Brownovo gibanje $\{B_t\}_{t \geq 0}$ je slučajen proces z naslednjimi lastnostmi: 
        \begin{enumerate}
            \item $B_0 = 0$.
            \item Prirastki $B_{t_n} - B_{t_{n-1}}, B_{t_{n-1}} - B_{t_{n-2}}, \ldots, B_2 - B_1, B_1 - B_0$ so neodvisne slučajne spremenljivke, za vsak $t_0 \leq t_1 \leq \cdots \leq t_{n-1} \leq t_n$.
            \item Za vsak $t \geq 0$ in $h > 0$ velja $B_{t+h} - B_t \sim \mathcal{N}(0, h)$.
            \item Funkcija $t \mapsto B_t$ je zvezna skoraj gotovo.
        \end{enumerate}
    \end{definicija}
    % Konec definicije
    
    Preden zapišemo izrek, definirajmo še pojem časa ustavljanja.
    
    % Začetek definicije
    \begin{definicija}
    Slučajna spremenljivka $\tau$ na verjetnostnem prostoru ($\Omega$, $\f$, $\mathsf{P}$) z vrednostmi v ??
    je čas ustavljanja glede na filtracijo $(\f_t)_{t \in T}$, če velja $\forall t \in T: \{\tau \leq t\} \in \f_t$.
    \end{definicija}
    % Konec definicije
    
    Zdaj lahko zapišemo izrek \ref{thm:stopped_brownian}. % Sklic na izrek z oznako thm:stopped_brownian
    
    % Začetek izreka
    \begin{izrek}
        \label{thm:stopped_brownian}
    Naj bo $\{B_t\}_{t \geq 0}$?? (standardno) Brownovo gibanje, $\tau$ čas ustavljanja glede na 
    $(\f_t)_{t \geq T}$ in naj velja $\mathsf{P} [\tau < \infty]=1$.
    Potem je tudi proces:
    \[
    \hat{B} := \{B_{T+t} - B_T \mid t \geq 0\}
    \]
    (standardno) Brownovo gibanje in neodvisen od $\f_T$.
    \end{izrek}
    % Konec izreka
  
    \bibliographystyle{plain}
    \bibliography{knjiga}

\end{document}